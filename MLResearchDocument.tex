% --------------------------------------------------------------
% This is all preamble stuff that you don't have to worry about.
% Head down to where it says "Start here"
% --------------------------------------------------------------
\documentclass[10pt]{article}
\usepackage[margin=1in]{geometry}
\usepackage{amsmath,amsthm,amssymb}
\usepackage{graphicx}
\usepackage{caption}
\usepackage{subcaption}
\usepackage{float}
\usepackage{verbatim}
\usepackage{color}
\usepackage[toc,page]{appendix}



\newcommand{\N}{\mathbb{N}}
\newcommand{\Z}{\mathbb{Z}}
\newenvironment{theorem}[2][Theorem]{\begin{trivlist}
\item[\hskip \labelsep {\bfseries #1}\hskip \labelsep {\bfseries #2.}]}{\end{trivlist}}
\newenvironment{lemma}[2][Lemma]{\begin{trivlist}
\item[\hskip \labelsep {\bfseries #1}\hskip \labelsep {\bfseries #2.}]}{\end{trivlist}}
\newenvironment{exercise}[2][Exercise]{\begin{trivlist}
\item[\hskip \labelsep {\bfseries #1}\hskip \labelsep {\bfseries #2.}]}{\end{trivlist}}
\newenvironment{problem}[2][Problem]{\begin{trivlist}
\item[\hskip \labelsep {\bfseries #1}\hskip \labelsep {\bfseries #2.}]}{\end{trivlist}}
\newenvironment{question}[2][Question]{\begin{trivlist}
\item[\hskip \labelsep {\bfseries #1}\hskip \labelsep {\bfseries #2.}]}{\end{trivlist}}
\newenvironment{corollary}[2][Corollary]{\begin{trivlist}
\item[\hskip \labelsep {\bfseries #1}\hskip \labelsep {\bfseries #2.}]}{\end{trivlist}}
\begin{document}
% --------------------------------------------------------------
% Start here
% --------------------------------------------------------------
\title{Applications of Reinforcement Learning in Robotics and Optimal Control}%replace X with the appropriate number
\author{\\ %replace with your name
Kristopher L. Reynolds\\
Matthew C. Sheckells
\\} %if necessary, replace with your course title
\maketitle
\section{Introduction}
The fields of robotics and optimal control have benefitted from concepts of reinforcement learning (RL) \cite{kober_reinforcement_2013} \cite{kaelbling_reinforcement_1996}. From an optimal control standpoint, the objective is to minimize some cost analogously to how RL aims to maximize a reward.  Several robotic systems in  \cite{bhasin_reinforcement_2011} and \cite{hester_rtmba:_2012} have all successfully implemented this notion and shown that performance equalled and in some cases exceeded that of hand-tuned controllers. The advantage of RL is that it is robust to changes in a system, such as malfunctioning actuators, changes in models parameters (i.e. inertia), and external disturbances. 

DISCUSS REFERENCES MORE

\section{Q-learning}
EXPLAIN GENERAL Q-LEARNING HERE

A variant of the algorithm, called Delayed Q-learning, has been shown to by Probably Approximately Correct in Markov Decision Processes (PAC-MDP) \cite{strehl_pac_2006}.  A PAC-MDP algorithm follows an $\epsilon$-optimal policy on all but a polynomial number of timesteps, with probability at least $1-\delta$ for $\epsilon>0, \delta > 0$.

\subsection{Delayed Q-learning}
Delayed Q-learning differs from traditional Q-learning in a few ways.  Rather than updating the Q-values at every time step, the algorithm updates $Q(s,a)$ when a state-action pair $(s,a)$ has been executed $m$ times since its last update.  It also uses a different update function:
$$Q_{t+1}(s,a) = \frac{1}{m}\sum_{i=1}^m(r_{k_i}+\gamma V_{k_i}(s_{k_i})) + \epsilon_1$$  
where $r_i$ is the $i$th reward received and $s_{k_1},\dots,s_{k_m}$ are the $m$ most recent next-states observed form executing $(s,a)$ at times $k_1 < \dots < k_m = t$, respectively. The user defines $\epsilon_1$ and $m$.   
% We will have some of these vars defined from earlier so I want define them all here
After $m$ executions of $(s,a)$ an update will only occur if the new Q-value estimate is at least $\epsilon_1$ smaller than the previous estimate.  If it is not smaller, then updates of $Q(s,a)$ are no longer allowed until another Q-value estimate (for a different state-action pair) is updated.

Waiting for $m$ samples before updating has an averaging effect that helps minimize any randomness. Adding $\epsilon_1$ to the update function helps achieve optimism which in turn facilitates safe exploration. By starting with high initial Q-values and only allowing Q-values to decrease ensures that each state is regularly visited and that each action is tried from every state.  This makes sure each state-action pair receives enough samples to converge to the correct Q-value.%<- discuss optimism more?
In the following section, we outline the proof that shows Delayed Q-learning is PAC-MDP.


\subsection{Proof Sketch: Delayed Q-learning is PAC-MDP}

Should probably explain how PAC-MDP is different than PAC?


The number of successful Q-value updates for a particular $(s,a)$ is bounded by

$$\kappa = \frac{1}{(1-\gamma)\epsilon_1}.$$

This follows from the fact that $Q(s,a)$ is initialized to $1/(1-\gamma)$ and that every $Q(s,a)$ results in a decrease of at least $\epsilon_1$ (and it is assumed rewards are non-negative).  For $A$ total actions and $S$ total states, at most $SA\kappa$ updates can occur.  Now, a state-action pair $(s,a)$ is initially allowed to attempt an update.  Then, a successful update of at most one other Q-value must occur  for $(s,a)$ to attempt another update.  So, there can be at most $1+SA\kappa$ attempted updates of $(s,a)$.  Therefore, there are at most
$$SA(1+SA\kappa)$$
total attempted updates.

During timestep $t$ of learning define $K_t$ to be the set of all state-action pairs $(s,a)$ such that 

$$Q_t(s,a) - \left(R(s,a)+\gamma\sum_{s'}T(s'|s,a)V_t(s')\right) \leq 3\epsilon_1.$$

A value for $m$ is specified as 

$$m=\frac{\ln{3SA(1+SA\kappa)/\delta}}{2\epsilon_1^2(1-\gamma)^2}.$$

Strehl et al. prove the following lemmas:

\begin{enumerate}

\item Consider the following assumption:  Suppose an attempted update of state-action pair $(s,a)$ occurs at time $t$, and that the $m$ most recent experiences of $(s,a)$ happened at times $k_1 < \dots < k_m = t$.  If $(s,a) \notin K_{k_1}$, then the attempted update will be successful.  The probability that this assumption is violated is at most $\delta/3$.  

% A proof sketch is given for this in Strehl

\item During the execution of Delayed Q-learning, $Q_t(s,a) \geq Q^*(s,a)$ holds for all timesteps $t$ and state-action pairs $(s,a)$ with probability at least $1-\delta/3$.

\item The number of timesteps $t$ such that a state-action pair $(s,a) \notin K_t$ is experienced is at most $2mSA\kappa$.

\item Let $M$ be an MDP, $K$ a set of state-action pairs, $M'$ an MDP equal to $M$ on $K$ (identical transition and reward functions), $\pi$ a policy, and
$T$ some positive integer. Let $A_M$ be the event that a state-action pair not in $K$ is encountered in a trial generated by starting from state $s$ and following $\pi$ for $T$ timesteps in $M$. Then,
$$V_M^\pi(s,T)\geq V_{M'}^{\pi}(s,T)-\text{Pr}(A_M)/(1-\gamma).$$

\end{enumerate}

Suppose Delayed Q-learning is run on an MDP $M$. It is assumed that the assumption in lemma 1 holds and that $Q_t(s,a) \geq Q^*(s,a)$.  These assumptions are broken with probability at most $2\delta/3$ by lemmas 1 and 2.  
Let $T=O\left(\frac{1}{1-\gamma}\ln{\frac{1}{\epsilon_2(1-\gamma)}}\right)$ be large enough so that $|V_{M'}^{\pi_t}(s_t,T) - V_{M'}^{\pi_t}(s_t)| \leq \epsilon_2$.   Let $\text{Pr}(U)$ denote the probability of
the algorithm performing a successful update on some
state-action pair $(s, a)$, while executing policy
$\mathcal{A}_t$ from state $s_t$ in $M$ for $T$ timesteps. From the previous statements and lemma 5 we have 
\begin{align*}
V_M^{\mathcal{A}_t} & \geq V_{M'}^{\mathcal{A}_t}(s_t,T)-\text{Pr}(A_M)/(1-\gamma) \\
 & \geq V_{M'}^{\pi_t}(s_t,T)-\text{Pr}(A_M)/(1-\gamma) -\text{Pr}(U)/(1-\gamma) \\
 & \geq V_{M'}^{\pi_t}(s_t)-\epsilon_2-(\text{Pr}(A_M)/(1-\gamma) +\text{Pr}(U)/(1-\gamma)). 
\end{align*}



More steps to be explained in between....

The number of timesteps needed for 
$$V_{\mathcal{A}_t}^{\pi_t}(s_t,T) \geq V^*(s_t)-\epsilon$$
is 
$$O\left(\frac{SA}{\epsilon^4(1-\gamma)^8}\ln{\frac{1}{\delta}}\ln{\frac{1}{\epsilon(1-\gamma)}}\ln{\frac{SA}{\delta\epsilon(1-\gamma)}}\right)$$
So, Delayed Q-learning will follow an $\epsilon$-optimal policy on all but $O\left(\frac{SA}{\epsilon^4(1-\gamma)^8}\ln{\frac{1}{\delta}}\ln{\frac{1}{\epsilon(1-\gamma)}}\ln{\frac{SA}{\delta\epsilon(1-\gamma)}}\right)$ steps with probability at least $1-\delta$.  Thus, it is PAC-MDP.
\\ 
\\
$R_{max}$ is another reinforcement learning algorithm proved to be PAC-MDP that we can discuss.

\nocite{yang_multiagent_2004}
\nocite{kim_autonomous_2003}

\newpage
\bibliography{ML}{}
\bibliographystyle{plain}




% --------------------------------------------------------------
% You don't have to mess with anything below this line.
% --------------------------------------------------------------
\end{document}